\documentclass[../main.tex]{subfiles}

\begin{document}
\chapter*{Abstract}
	%Die Arbeit \glqq Studnetz, eine Applikation für Mobilgeräte zur Vermittlung von Nachhilfeunterricht\grqq{} befasst sich mit der Entwicklung einer Applikation für Android-Mobilgeräte. Dabei wird vertieft auf die Funktion der einzelnen Komponenten der Applikation eingegangen. Anhand konkreter Beispiele aus der Applikation werden die umgesetzten Lösungsansätze für auftretende Herausforderungen genauer beschrieben, um so die Funktionsweise der Applikation ersichtlich zu machen.
	
	%Das Ziel ist es, die Vermittlung von Nachhilfe unter Schülerinnen und Schülern zu vereinfachen. In der Applikation ist es Schülerinnen und Schülern möglich Fächer anzugeben, in welchen sie andere unterstützen können. Des Weiteren können sie nach Schülerinnen und Schülern suchen, die ihnen in bestimmten Fächern helfen können. Dabei kann nach Klasse, Schule und Fächer gefiltert werden. Durch einen Chat wird eine Kommunikation zwischen den Schülern und Schülerinnen ermöglicht, wo genauere Daten wie Datum, Zeit und Ort für ein Treffen ausgetauscht werden können.
	
	%Die Applikation basiert auf einem Client-Server-Konzept. Der Client wurde dabei in der Programmiersprache \emph{Java} und der Auszeichnungssprache \emph{XML} programmiert. Auf der Seite der Server findet sich zum Einen ein Webserver mit einer installierten \emph{MySQL}-Datenbank und einer Reihe von Skripten in der Programmiersprache \emph{PHP}, welcher für das Speichern der einzelnen Benutzer verantwortlich ist. Zum Anderen wird ein \emph{Firebase}-Server verwendet, welcher eine Chatfunktion in Echtzeit ermöglicht.
	
	%\glqq Studnetz\grqq{} ist ein Prototyp für eine potentiell marktfähige Applikation. Sie kann zwar 
	
	%\section{new}
	%Die Arbeit \glqq Studnetz, eine Applikation für Mobilgeräte zur Vermittlung von Nachhilfeunterricht\grqq{} befasst sich mit der Entwicklung einer Applikation für Android-Mobilgeräte. Dabei wird vertieft auf die Funktionsweise der entwickelten Applikation eingegangen. Hierzu werden zuerst Schritt für Schritt die einzelnen Komponenten und ihre Funktionsweise erläutert, um anschliessend ihre Rolle in der entwickelten Applikation zu beschreiben.
	
	
	Die Arbeit \glqq Studnetz, eine Applikation für Mobilgeräte zur Vermittlung von Nachhilfeunterricht\grqq{} befasst sich mit der Entwicklung einer Applikation für Android-Mobilgeräte. Hierzu werden zuerst Schritt für Schritt die einzelnen Komponenten erläutert, um anschliessend ihre Rolle in der entwickelten Applikation zu beschreiben. Das Ziel ist es, den Lesern und Leserinnnen dieser Arbeit ein tieferes Verständnis für die Funktionsweise und die aufgetretenen Herausforderungen der bei der Entwicklung zu vermitteln.
	
	Studnetz selber ist einer Applikation, die die Vermittlung von Nachhilfe unter Schülerinnen und Schülern zu vereinfachen soll. In der Applikation ist es Schülerinnen und Schülern möglich Fächer anzugeben, in welchen sie andere unterstützen können. Des Weiteren können sie nach Schülerinnen und Schülern suchen, die ihnen in bestimmten Fächern helfen können. Dabei kann nach Klasse, Schule und Fächer gefiltert werden. Durch einen Chat wird eine Kommunikation zwischen den Schülern und Schülerinnen ermöglicht, wo genauere Daten wie Datum, Zeit und Ort für ein Treffen ausgetauscht werden können.
	
	Die Applikation basiert auf einem Client-Server-Prinzip. Der Client wurde dabei in der Programmiersprache \emph{Java} und der Auszeichnungssprache \emph{XML} programmiert. Auf der Seite der Server findet sich zum Einen ein Webserver mit einer installierten \emph{MySQL}-Datenbank und einer Reihe von Skripten in der Programmiersprache \emph{PHP}, welcher für das Speichern der einzelnen Benutzer verantwortlich ist. Zum Anderen wird ein \emph{Firebase}-Server verwendet, welcher eine Chatfunktion in Echtzeit ermöglicht.
	
	Das Endprodukt dieser Arbeit ist ein solider Prototyp für eine potenziell Marktfähige Applikation. Mit noch etwas Aufwand könnte die Studnetz-Applikation schon in naher Zukunft der Öffentlichkeit zugänglich gemacht werden.
	
\end{document}