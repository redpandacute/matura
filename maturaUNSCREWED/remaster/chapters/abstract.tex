\documentclass[../main.tex]{subfiles}

\begin{document}
\chapter*{Abstract}
	Die Arbeit \glqq Studnetz, eine Applikation für Mobilgeräte zur Vermittlung von Nachhilfeunterricht\grqq{}  befasst sich mit der Entwicklung einer Applikation für Android-Mobilgeräte. Es wird vertieft auf die Funktion der einzelnen Komponenten der Applikation eingegangen. Es werden mithilfe von konkreten Beispielen aus der Applikation die umgesetzten Lösungsansätze für auftretende Herausforderungen genauer beschrieben um so die genaue Funktionsweise der Applikation ersichtlich zu machen.
	
	Die Applikation \glqq Studnetz\grqq{} ist ein Prototyp für eine potentiell marktfähige Applikation für Mobilgeräte mit dem Betriebssystem Android. Das Ziel der Applikation ist die Vermittlung von Nachhilfe unter Schülerinnen und Schülern zu vereinfachen. In der Applikation ist es Schülerinnen und Schülern möglich, Fächer anzugeben, in welchen sie andere unterstützen können. Desweiteren können sie nach Schülerinnen und Schülern suchen, die ihnen in bestimmten Fächern helfen können. Dabei kann nach Klasse, Schule und Fächer gefiltert werden. Durch einen Chat wird eine Kommunikation zwischen den Schülern und Schülerinnen ermöglicht, wo genauere Daten wie Datum, Zeit und Ort ausgetauscht werden können.
	
	Die Applikation basiert auf einem Client-Server-Konzept. Der Client wurde dabei in der Programmiersprache \emph{Java} und der Auszeichnungssprache \emph{XML} programmiert. Auf der Seite der Server findet sich ein Webserver mit einer installierten \emph{MySQL}-Datenbank und einer Reihe von Skripten in der Programmiersprache \emph{PHP}, welcher für das Speichern der einzelnen Benutzer verantwortlich ist. Des Weiteren wird ein \emph{Firebase}-Server verwendet, welcher eine Chatfunktion in Echtzeit ermöglicht.

\end{document}