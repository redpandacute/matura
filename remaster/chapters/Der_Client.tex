\documentclass[../main.tex]{subfiles}

\begin{document}
	\chapter{Der Client}
	Der Client ist das Herzstück der entwickelten Anwendung. Er ist der Teil der Anwendung, welcher von den Endbenutzern/Endbenutzerinnen heruntergeladen und benutzt wird. Er ist das verbindende Glied zwischen den gespeicherten Informationen auf dem Server und den Benutzern/Benutzerinnen. Im folgenden Kapitel wird genauer auf die Funktionsweise des Clients eingegangen und seine Interaktionen mit dem Server beschrieben.
	
	\section{Android}
	Der Client wurde für das Betriebssystem \emph{Android} entwickelt. Es wird geschätzt, dass zwischen 85 und 86 Prozent aller heute verwendeten Mobiltelefone eine Version des Betriebssystems Android installiert haben. Dies macht Android zum mit Abstand meist verwendeten Betriebssystem für Mobiltelefone weltweit. Die Entwicklung der Studnetz Applikation für Android Geräte macht also nur Sinn, da somit eine grosse Menge an Benutzern/Benutzerinnen erreicht werden kann. \cite{android}
	
	Android basiert auf einem stark modifizierten Linux Kernel. Programme werden darauf in einer sogenannten \emph{Android Runtime} Umgebung ausgeführt. Diese Umgebung ist in der Lage, Bytecodes im \emph{Dex-Format} (Dalvik Executable-Format) auszuführen. Die Bytecodes werden dabei aus von der Programmiersprache \emph{Java} erstellten Bytecodes generiert. Die meisten Applikationen für Android wurden deshalb lange in der Programmiersprache Java programmiert, bis vor wenigen Jahren langsam ein Umschwung zur neueren Programmiersprache \emph{Kotlin} wahrnehmbar wird. Kotlin benutzt für die Ausführung das gleiche Bytecode-Format wie Java, bietet jedoch für die Entwicklung, besonders in den Bereichen Syntax, Vorteile gegenüber Java. \cite{androidJava}
	
	\subsection{Java}
	Für die Entwicklung der Studnetz Applikation wurde die Programmiersprache Java gewählt, da sie sehr gut dokumentiert ist und ich, als Entwickler, bereits vor dieser Arbeit Erfahrungen mit Java gesammelt hatte. Java ist eine Sprache der 3. Generation und gehört zu den objektorientierten Programmiersprachen. Sie wurde erstmals 1995 von Sun Microsystems veröffentlicht und ist seit 2010 in Besitz von Oracle. Java zeichnet sich besonders durch seine plattformunabhängigkeit aus und eignet sich daher gut für kleinere Applikationen, die auf vielen verschiedenen Geräten funktionieren sollen. Java wird innerhalb des Clients hauptsächlich für die logischen Abläufe hinter der Benutzeroberfläche verwendet. 
	
	\subsection{XML}
	\subsection{Android Studio}
	
	
	\section{Architektur des Clients}
	\subsection{Klassenübersicht}
	\subsubsection{Activities}
	\subsubsection{Fragments}
	\subsubsection{Models}
	\subsubsection{Requests}
	\subsubsection{Adapter und ViewHolder}
	\subsubsection{Utility-Klassen}
	
	\subsection{Beispielhafte Erläuterung der \emph{MainActivity}}
	\subsubsection{Aufbau der Activity}
	\subsubsection{Aufbau der Fragments}
	
	\subsection{Interaktion mit dem Webserver (MySQL Datenbank)}
	\subsection{Interaktion mit der Firebase Echtzeitdatenbank}
	
\end{document}