\documentclass[../main.tex]{subfiles}

\begin{document}
\section*{Vorwort}
Diese Arbeit befasst sich mit der Entwicklung einer Applikation für Mobiltelefone. Die Applikation soll die Vermittlung von Nachhilfeunterricht unter Schülerinnen und Schülern einer Schule vereinfachen. Dabei ist anzumerken, dass die Entwicklung einer vollständig markttauglichen Applikation den Rahmen dieser Arbeit sprengen würde. Das Ziel ist vielmehr das Legen eines Grundsteins, aus welchem ein solches Produkt entstehen könnte.

\subsubsection*{Zu mir}
Ich möchte an dieser Stelle die Gelegenheit ergreifen, ein paar Worte zu mir, dem Autor und Entwickler dieser Arbeit, zu verlieren.
	
Zum Zeitpunkt, an welchem ich begann die Studnetz-Applikation zu entwickeln, hatte ich zuvor noch nie eine Applikation für Mobiltelefone entwickelt. Bis zu diesem Zeitpunkt habe ich mich hauptsächlich auf die Entwicklung kleinerer Programme für Computer beschränkt. Das dazu nötige Wissen habe ich mir grösstenteils selber angeeignet. So hatte ich mich vor diesem Projekt noch nie mit Datenbanken und Webdevelopement befasst. Ich habe mir dieses Projekt als Herausforderung genommen, um etwas Neues zu lernen und mich als Entwickler weiter zu entwickeln. Dieser Lernprozess hat definitiv stattgefunden. Dies erklärt auch, weshalb teilweise gleiche Probleme an verschiedenen Orten verschieden gelöst wurden oder warum gewisse verwendete Lösungen von anerkannten Praktiken abweichen.

\subsubsection*{Motivation}
Die Motivation für die Entwicklung der Studnetz-Applikation schöpfte ich haupsächlich aus zwei Quellen. Zum einen war es der Wille, neue Teilgebiete des Programmierens kennenzulernen und zu verstehen, weshalb die Wahl dann sehr schnell auf eine Applikation für Mobiltelefone fiel. Immerhin sind Mobiltelefone in der heutigen Welt kaum mehr wegzudenken und die Fähigkeit, Programme für sie zu entwickeln hat mich schon lange fasziniert. Die zweite Motivation ist dann aus einer Idee von Herrn Bättig entstanden, der während der einführenden Präsentation in die Maturarbeit eine Applikation zur Vermittlung für Nachhilfe erwähnte. Zuerst bin  ich skeptisch gewesen, doch schnell habe ich im Gespräch mit anderen festgestellt, dass eine gut entwickelte Applikation in diesem Bereich durchaus Verwendung und Beliebtheit finden könnte. Dies ist dann der Anstoss gewesen, ein solches Projekt in Angriff zu nehmen. 
	
\subsubsection*{Danksagungen}
Ich möchte mich hier bei den Personen bedanken, die mich während dieser Arbeit unterstützt haben. Zuerst einmal möchte ich meiner betreudenen Lehrperson Andreas Umbach danken. Er stellte mir nicht nur einen Platz auf der Datanbank des Ergänzungsfaches zur Verfügugn, sondern half mir stets bei allfälligen Fragen und Unklarheiten weiter. Zudem möchte ich ein Dankeschön an  meinen Onkel Peter Arrenbrecht richten, der mir mit seinem Hinweis auf die Firebase API komplett neue Möglichkeiten eröffnet hat. Als letztes gilt noch ein gigantisches Dankeschön an meinen Vater, der mich während dieser ganzen Arbeit als Ratgeber und Korrekturleser immer unterstützte und immer ein offenes Ohr für meine Probleme zeigte.
\end{document}