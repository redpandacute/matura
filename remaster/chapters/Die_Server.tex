\documentclass[../main.tex]{subfiles}

\begin{document}
	\chapter{Die Server}
	Im folgenden Kapitel wird auf die beiden Server eingegangen, welche für die Studnetz Applikation verwendet werden. Es soll beschrieben werden, wie mit den Daten auf den Servern umgegangen wird und in wie fern dies mit dem Client zusammenhängt. Ein schematischer Überblick über das Verhältnis der beiden Server zum Client findet sich in Abbildung \ref{ServerClientSchema}. Dabei fällt auf, dass anders als in Kapitel \ref{ClientServerPrinzip} beschrieben, dass der Firebase Server sich nicht rein passiv verhält, sondern durchaus aktiv mit dem Client interagiert. Dazu mehr in Kapitel \ref{DieFirebaseDatenbank}.
	
	\section{Webserver}
	Bei den beiden in der Applikation verwendeten Server handelt es sich genau genommen und sogenannte Webserver. Ein Webserver beschreibt eine Software, welche das Abrufen von lokalen Diensten (z.B. Programme oder Skripte) und gespeicherten Daten über das Internet ermöglicht. Clients sind dann in der Lage, über die IP-Adresse des Webservers die bereitgestellten Dienste und Daten aufzurufen. Der Webserver ist ebenfalls in der Lage, Antworten auf Anfragen der Clients zurückzusenden.
	
	Es sind genau solche Skripte, die einen Zugriff auf eine sich ebenfalls auf dem Server befindende Datenbank ermöglichen. Dafür beliebte Skriptsprachen sind Sprachen wie PHP (siehe Kapitel \ref{php}), Ruby, Python oder Pearl. Diese Skripte sind in der Lage, Anfragen an die Datenbank zu stellen und Antworten zu formulieren, welche dann vom Webserver an den Client zurückgeschickt werden können. Eine direkte Kommunikation zwischen Datenbank und Client findet auf einem Webserver so gut wie nie statt.
	
	\subsection{PHP}\label{php}
	PHP ist eine Open-Source Skriptsprache, welche grosse Beliebtheit in der Server seitigen Webentwicklung findet. Die Abkürzung PHP ist ein rekursives Akronym für \emph{Hyptertext Preprocessor} und wurde speziell für die Webprogrammierung entwickelt. Sich auf Webservern befindende PHP Skripte bieten den grossen Vorteil, dass sie jeweils nur auf dem Server ausgeführt werden. Somit ist es Clients zwar möglich die Skripte via Webadresse auf einem Webserver auszuführen, bekommen jedoch die Codestruktur des PHP Skriptes dabei nie zu Gesicht. \cite{PHP} Im Skript selber wird dann oftmals eine Antwort formuliert, welche dann vom Webserver an den Client zurückgeschickt werden kann. Die Form der Antwort unterscheidet sich dabei stark von Anwendung zu Anwendung. Antworten können in Sprachen wie HTML, JSON oder JavaScript verfasst sein, wobei jedoch auch Bild- und PDF-Dateien durchaus als Antwortformate in Frage kommen.\cite{PHP:function}
	
	\subsection{JSON}
	JSON ist eine Abkürzung für \emph{JavaScript Object Notation} und ist eine Datenformat, welches für den Austausch von strukturierten Daten entwickelt wurde. Bei der Entwicklung wurde dabei besonders auf drei Kriterien geachtet: Leserlichkeit für Menschen sowie einfaches Generieren sowie Parsen auf Seiten des Computers. Diese sehr gut umgesetzten Eigenschaften und seine plattformunabhängigkeit  machen JSON zu einem äusserst effizienten Datenaustauschformat für einfachere Datentypen und wird besonders in der Webentwicklung gerne verwendet. \cite{JSON}
	
	\section{Datenbanken} \label{Datenbanken}
	Eine der wohl wichtigsten Aufgaben von Computern ist das Speichern, Verwalten und auch Manipulieren von Informationen. Anwendungen, die sich hauptsächlich mit dieser Aufgabe beschäftigen werden allgemein als \emph{Datenbanken} bezeichnet. Sie haben die Aufgabe, Informationen systematisch zu ordnen, zu speichern und bei Bedarf zu verändern. Grundsätzlich bezeichnet der Begriff Datenbank gleich zwei Dinge auf einmal. Zum einen wird ein strukturierter Speicher von Informationen als Datenbank bezeichnet und zum anderen jedoch auch die Anwendung, die das Verwalten der Daten überhaupt erst ermöglicht. Solche Anwendungen werden genauer als \emph{Database Management System} (DBMS) bezeichnet und sind meist hochkomplex in ihren Funktionsweisen, um die effiziente Verwaltung von selbst riesigen Datenmengen zu ermöglichen. \cite{IT-Handbuch} Datenbanken kommen oftmals auf zentralen Servern zum Einsatz, wo zum Beispiel die Benutzer eines Webdienstes oder die Bestellungen einer Firma aufgelistet werden. Sie stellen den dynamischen Speicher eines Servers bzw. Webservers dar.
	
	Die Informationen in einer Datenbank werden meist in einer auf Tabellen basierenden Struktur gespeichert. Dabei wird eine einzelnen Zeile in der Tabelle als Datensatz oder Eintrag bezeichnet, während die verschiedenen Spalten Felder genannt werden. Beim Erstellen einer solcher Tabelle werden zuerst die verschiedenen Felder bestimmt. Ihnen wird ein Name gegeben, der beschreibt, was darin gespeichert werden soll. Hinzu kommt ein Datentyp, der angibt um was es sich bei diesem Feld handelt (z.B. eine Zahl oder einen kurzen Text). Es ist ebenfalls möglich einem Feld einen Default Wert zuzuschreiben. Dieser wird dann für einen Datensatz verwendet, wenn das Feld sonst nicht definiert wurde. Zuletzt ist es noch möglich, einem Feld speziellere Eigenschaften zuzuschreiben. Dazu gehört unter anderem eine Funktion mit dem Namen \emph{auto\_increment}. Sie bewirkt, dass im ihr zugeschriebenen Feld einem neu eingetragenen Datensatz automatisch ein in der Tabelle einzigartiger Wert des Typen Integer (Datentyp für natürliche Zahlen) zugewiesen wird.
	
	Datenbanken selber können in verschiedene Typen eingeteilt werden, die alle ihre eigenen Vor- und Nachteile mit sich bringen. Die einfachste Form eines Datenbanktyps ist wohl die \emph{Einzeltabellendatenbank}. Sie besteht aus nur einer Tabelle, in welcher alle Informationen abgespeichert werden. Sie eignet sich gut für kleine, übersichtliche Tabellenstrukturen wie zum Beispiel eine einfache Liste von Adressen. Die Einzeltabellendatenbank stösst jedoch spätestens dann an ihrer Grenzen, wenn die Informationen nicht mehr in nur einer, sondern gleich mehreren Tabellen gespeichert werden sollen. Hier tritt ein anderer Datenbanktyp ins Spiel. Eine \emph{relationale Datenbank} ist in der Lage, verschiedene Tabellen logisch miteinander zu verknüpfen und sich darin zu orientieren. Diese logische Verknüpfung ist möglich aufgrund eindeutiger Eigenschaften eines Datensatzes. Dies kann zum Beispiel eine Kundennummer oder ein Name sein. Ein solches Feld wird auch als ein \emph{Key} bezeichnet. Wichtig dabei ist, dass jeder Key nur einmal in einer Tabelle vorkommt, da ansonsten keine eindeutige Verknüpfungen möglich sind. Die Anwendung zur Verwaltung einer solchen relationalen Datenbank wird \emph{Relational Database Management System} (RDBMS) genannt. \cite[S. 745 - 751]{IT-Handbuch}
	
	\section{Die MySQL Datenbank}
	Eines der am weitesten verbreiteten RDBMS ist die MySQL Datenbank. Das System wurde ursprünglich von den drei Gründern Allan Larsson, Michael Widenius und David Axmark 1995 entwickelt, wurde später von \emph{Sun Microsytems} aufgekauft und gelangte schlussendlich in den Besitz des amerikanischen Softwareherstellers \emph{Oracle}. MySQL ist unter einem dualen Lizenzsystem eingetragen, sodass die Software zum einen unter einer \emph{General Public Licence} (GPL) \cite{GPL}, aber auch unter eine proprietäre Lizenz \cite{proprietäreLizenz} gestellt ist. \cite{tecmint.com} Das MySQL System darf aufgrund der GPL gratis heruntergeladen, installiert und modifiziert werden.
	
	MySQL Datenbanken sind besonders bei der Betreibung von Webdiensten aller Art von grosser Beliebtheit. Sie befinden sich dann, wie bereits in Kapitel \ref{Datenbanken} erwähnt, auf einem zentralen Server. Mithilfe sogenannter \emph{Queries} (Datenbank Abfragen) ist es dem Server möglich mit der Datenbank zu interagieren. Solche Queries sind im Falle einer MySQL Datenbank in der Datenbanksprache \emph{SQL} (Structured Query Language) formuliert. Queries können in vier Arten von Abfragen unterteilt werden: \cite[S. 760]{IT-Handbuch}
	
	\begin{itemize}
		\item Auswahlabfragen (\emph{Select Queries}) geben den Inhalt von einem oder mehreren Feldern aus einer oder verschiedenen Tabellen zurück. Dabei kann bei Bedarf nach Kriterien gefiltert werden, um die Suche nach bestimmten Datensätzen einzugrenzen.\cite[S. 746]{IT-Handbuch}
		\item Einfügeabfragen (\emph{Insert Queries}) fügen einen neuen Datensatz zu einer bestehenden Tabelle hinzu.\cite[S. 746]{IT-Handbuch}
		\item Änderungsabfragen (\emph{Update Queries}) ändern bestimmte oder alle Felder eines bestehenden Datensatzes in einer Tabelle.\cite[S. 746]{IT-Handbuch}
		\item Löschabfragen (\emph{Delete Queries}) löschen einen Datensatz aus einer Tabelle. \cite[S. 746]{IT-Handbuch}
	\end{itemize}
	
	\section{Firebase} \label{DieFirebaseDatenbank}
	Firebase ist eine  Entwicklungsplattform für Webapplikationen, welche seit 2014 von Google angeboten wird. Firebase entwickelte sich aus dem 2011 gegründete Startup \emph{Envolve} der beiden Gründern James Tamplin und Andrew Lee. Das Ziel von Envolve war es, Kunden eine API (\emph{Application Programming Interface}) zu bieten, mit welcher in Echtzeit synchronisierte Chatfunktionen einfach realisiert werden können. Nachdem jedoch viele Benutzer die API für andere Anwendungen als Chats verwendeten, ja sogar einzelne Spielentwickler sie für eine Realtime Synchronisation verschiedener Clients verwendeten, begann die Entwicklung sich auf das Anbieten einer API für Echtzeitsysteme zu konzentrieren. Der Erfolg war gross und 2014 wurde das Unternehmen von Google aufgekauft. Google entwickelte aus Envolve daraufhin eine Plattform, welche verschiedenste Tools zur Webdienstentwicklung beinhaltet und heute unter dem Namen Firebase vermarktet wird. Firebase bietet sowohl Tools für die Entwicklung neuer Dienste wie Echtzeitdatenbanken, Authentifizierungsfunktionen und Crashanalysen, aber auch für das Unterhalten von bestehenden Diensten. Beispiele für ein solche Tools zur Unterhaltung bestehender Dienste wären zum Beispiel das Senden von Push-Benachrichtigungen an alle Clients oder das Überwachen von geschalteter Werbung innerhalb der Clients. Die Tools dürfen in einem begrenzten Rahmen gratis verwendet werden und sind daher sehr attraktiv für kleinere Entwicklerunternehmen und Lernende, eignen sich aber durchaus auch für grössere Unternehmen.\cite{Firebase}
	
	\subsection{Die Firebase Echtzeitdatenbank}
	Wie bereits erwähnt, bietet Firebase unter anderem auch eine sogenannte Echtzeitdatenbank (Realtime Database) an. Diese Echzeitdatenbank ist eine NoSQL Datenbank und unterscheidet sich in ihrer Funktionsweise stark von traditionellen Datenbanken wie MySQL oder MariaDB. Anders als traditionelle Datenbanken auf Webservern agieren Echtzeitdatenbanken nicht nur passiv auf Anfragen, sondern teilen den Clients aktiv mit, wenn sich ein Datensatz verändert hat (Push-Based Data Access) und halten sie so immer auf dem neusten Stand. Solche Mitteilungen über Datensatzänderungen bezeichnet mal allgemein als \emph{Pushes}. Echtzeitdatenbanken werden besonders dann eingesetzt, wenn sich ein Datensatz häufig oder jederzeit ändern kann und es notwendig ist, dass die Clients ohne grosse Verzögerung davon unterrichtet werden.\cite{RealtimeDatabase} Bei der Firebase Echtzeitdatenbank ist es nicht einmal nötig, dass die Clients zur Zeit des Pushes online sind. Sie werden beim nächsten Start automatisch dann mit dem Datensatz auf der Datenbank abgeglichen und aktualisiert.\cite{FirebaseRTDB}
	
	\section{Implementation in der entwickelten Applikation}
	
	\subsection{Datenbankstrukturen}
	\subsubsection{MySQL Datenbankstruktur}
	\subsubsection{Firebase Datenbankstruktur}
	\subsection{PHP Skripte}
	
\end{document}