\documentclass[../main.tex]{subfiles}

\begin{document}
	\chapter{Diskussion}
	
	\section{Verbesserungspotenzial}
	In dem Stand, in welchem sich die entwickelte Applikation zur Zeit befindet, ist noch lange nicht an eine mögliche Veröffentlichung zu denken. Mögliche Mängel und nötige nächste Schritte sollen in diesem Kapitel etwas genauer diskutiert und betrachtet werden.
	
	\subsection*{Datenschutz}
	Damit eine Applikation heutzutage auf dem Markt angeboten werden darf, gelten eine Reihe von Gesetzen, die die Nutzer/Nutzerinnen einer Applikation schützen sollen.  Dieses Thema wurde mit der neuen DSGVO (Datenschutz Grundverordnung), welche in der EU seit 2018 gilt,  wieder neu Aufgefrischt und neue Regelungen eingeführt. Zwar gilt die DSGVO nicht für Schweizer und Schweizerinnen, jedoch für sämtliche EU-Bürger/EU-Bürgerinnen, deren Daten von der Applikation erfasst werden. Zudem kommt, dass sogar in der Schweiz in naher Zukunft Anpassungen im Datenschutz Gesetz (DSG) zu erwarten sind. Zu riskieren, die Applikation ohne Befolgung dieser Gesetze zu veröffentlichen, wäre also nahezu halsbrecherisch. Damit die Applikation regelkonform wäre, fehlt es ihr jedoch noch an einer Reihe von Features wie die Funktion, einen Account vollständig löschen zu können. Vor einem rechtlich abgesicherten Release müssten also noch eine Reihe von Änderungen und Erweiterungen vorgenommen werden.  \cite{dsgvoschweiz}
	
	\subsection*{Sicherheit}
	Die entwickelte Studnetz Applikation verfügt über noch sehr rudimentäre und teilweise sogar stark unzureichende Sicherheitsvorkehrungen. Diese müssten vor einem Release unbedingt behoben werden um unschöne Szenen zu vermeiden.
	
	\subsection*{Betatesting und Kompatibilität}
	Die Applikation ist für einen Release noch unzureichend Getestet. Dazu gehören sowohl Tests bezüglich ihrer Verwendung auf verschiedenen Geräten und Android Versionen wie auch Tests bezüglich ihrer Anwendbarkeit, wozu auch Betatest mit einer Gruppe von ausgewählten Benutzern gehörte. Besonders Betatests können durch Feedback von Benutzern einer Applikation extrem helfen, da dann oft Fehler und Mängel entdeckt werden, die den Entwicklern/Entwicklerinnen ansonsten entgangen wären.
	
	\subsection*{Benutzeroberfläche}
	Die Benutzeroberfläche ist das Verbindende Glied zwischen Funktion und Benutzer/Benutzerin. Wenn die Benutzeroberfläche einer Applikation die Benutzer/die Benutzerinnen nicht anspricht, werden sie auch die Funktionen einer Applikation kaum benutzen. Deshalb sind gut durchdachte Benutzeroberflächen heutzutage eine mindestens so wichtige Komponente einer Applikation wie ihre Funktionalität.
	
	Über die Qualität der Benutzeroberfläche der Studnetz Applikation lässt sich streiten. Sie ist definitiv nicht Perfekt, erfüllt jedoch ihren Zweck. In meinen Augen jedoch gibt es definitiv Teile der Applikation, welche ein visuelles Upgrade ertragen könnten. Gewisse Screens wirken zu überfüllt und die Icons für die Fächer wirken in ihrer Farbe etwas unpassend.
	
	\section{Fazit}
	Das Ziel dieser Arbeit war es, eine Applikation zu entwickeln, die in ihren Grundelementen die Vermittlung von Nachhilfe unter Schülerinnen und Schülern vereinfachen soll. Dieses Ziel konnte zu einem grossen Teil erreicht werden. Die Applikation verfügt über alle dazu nötigen Features und bildet ein solides Fundament für die Entwicklung einer markttauglichen Version.
	
	Weiter habe ich als Entwickler während dieser Arbeit mir sehr viel neues Wissen aneignen können. Ich habe viel gelernt über die Planung und Umsetzung von grösseren Programmierprojekten. Zudem kommt das ganze technische  Wissen, welches nötig gewesen ist, um dieses Projekt zu realisieren.
\end{document}